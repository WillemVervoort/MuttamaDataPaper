%% Copernicus Publications Manuscript Preparation Template for LaTeX Submissions
%% ---------------------------------
%% This template should be used for copernicus.cls
%% The class file and some style files are bundled in the Copernicus Latex Package, which can be downloaded from the different journal webpages.
%% For further assistance please contact Copernicus Publications at: production@copernicus.org
%% https://publications.copernicus.org/for_authors/manuscript_preparation.html

%% copernicus_rticles_template (flag for rticles template detection - do not remove!)

%% Please use the following documentclass and journal abbreviations for discussion papers and final revised papers.

%% 2-column papers and discussion papers
\documentclass[, manuscript]{copernicus}



%% Journal abbreviations (please use the same for preprints and final revised papers)

% Advances in Geosciences (adgeo)
% Advances in Radio Science (ars)
% Advances in Science and Research (asr)
% Advances in Statistical Climatology, Meteorology and Oceanography (ascmo)
% Aerosol Research (ar)
% Annales Geophysicae (angeo)
% Archives Animal Breeding (aab)
% Atmospheric Chemistry and Physics (acp)
% Atmospheric Measurement Techniques (amt)
% Biogeosciences (bg)
% Climate of the Past (cp)
% DEUQUA Special Publications (deuquasp)
% Earth Surface Dynamics (esurf)
% Earth System Dynamics (esd)
% Earth System Science Data (essd)
% E&G Quaternary Science Journal (egqsj)
% EGUsphere (egusphere) | This is only for EGUsphere preprints submitted without relation to an EGU journal.
% European Journal of Mineralogy (ejm)
% Fossil Record (fr)
% Geochronology (gchron)
% Geographica Helvetica (gh)
% Geoscience Communication (gc)
% Geoscientific Instrumentation, Methods and Data Systems (gi)
% Geoscientific Model Development (gmd)
% History of Geo- and Space Sciences (hgss)
% Hydrology and Earth System Sciences (hess)
% Journal of Bone and Joint Infection (jbji)
% Journal of Micropalaeontology (jm)
% Journal of Sensors and Sensor Systems (jsss)
% Magnetic Resonance (mr)
% Mechanical Sciences (ms)
% Natural Hazards and Earth System Sciences (nhess)
% Nonlinear Processes in Geophysics (npg)
% Ocean Science (os)
% Polarforschung - Journal of the German Society for Polar Research (polf)
% Primate Biology (pb)
% Proceedings of the International Association of Hydrological Sciences (piahs)
% Safety of Nuclear Waste Disposal (sand)
% Scientific Drilling (sd)
% SOIL (soil)
% Solid Earth (se)
% State of the Planet (sp)
% The Cryosphere (tc)
% Weather and Climate Dynamics (wcd)
% Web Ecology (we)
% Wind Energy Science (wes)

% Pandoc citation processing

% The "Technical instructions for LaTex" by Copernicus require _not_ to insert any additional packages.
% 
% tightlist command for lists without linebreak
\providecommand{\tightlist}{%
  \setlength{\itemsep}{0pt}\setlength{\parskip}{0pt}}


%%\usepackage{booktabs}
\usepackage{longtable}
\usepackage{array}
\usepackage{multirow}
\usepackage{wrapfig}
\usepackage{float}
\usepackage{colortbl}
\usepackage{pdflscape}
\usepackage{tabu}
\usepackage{threeparttable}
\usepackage{threeparttablex}
\usepackage[normalem]{ulem}
\usepackage{makecell}
\usepackage{xcolor}
%
%% \usepackage commands included in the copernicus.cls:
%\usepackage[german, english]{babel}
%\usepackage{tabularx}
%\usepackage{cancel}
%\usepackage{multirow}
%\usepackage{supertabular}
%\usepackage{algorithmic}
%\usepackage{algorithm}
%\usepackage{amsthm}
%\usepackage{float}
%\usepackage{subfig}
%\usepackage{rotating}

\begin{document}


\title{A detailed streamflow and groundwater salinity dataset for
Muttama Creek Catchment, NSW, Australia}


\Author[1]{R. Willem}{Vervoort}
\Author[1][floris.vanogtrop@sydney.edu.au]{Floris}{van Ogtrop}
\Author[1]{Mina}{Tambrchi}
\Author[1]{Farzina}{Akter}
\Author[1]{Alexander}{Buzacott}
\Author[1]{Jason}{Lessels}
\Author[1]{James}{Moloney}
\Author[1]{Dipangkar}{Kundu}
\Author[1]{Feike}{Dijkstra}
\Author[1]{Thomas}{Bishop}


\affil[1]{Sydney Institute of Agriculture, The University of Sydney, NSW
2006}

\runningtitle{Muttama Creek Salinity Data Set}

\runningauthor{Vervoort et al.}


\correspondence{Floris\ van Ogtrop\ (floris.vanogtrop@sydney.edu.au)}



\received{}
\pubdiscuss{} %% only important for two-stage journals
\revised{}
\accepted{}
\published{}

%% These dates will be inserted by Copernicus Publications during the typesetting process.


\firstpage{1}

\maketitle


\begin{abstract}
Dryland salinity remains a major global natural resource management
concern, and which is amplified in Australia. However, limited detailed
space-time data sets with observations of stream and groundwater
salinity has constrained a deep understanding of the range of processes
that can lead to dryland salinity problems in landscapes. The aim of
this study is to report on the open data set resulting from a 14-year
data collection effort in a subcatchment of the Murrumbidgee catchment
in New South Wales, Australia. Over a 14-year period a series of
different sampling campaigns has resulted in a large dataset with
hydrogeochemical data which includes both in-situ (field) data and post
laboratory analysis of major anions and cations. This data is augmented
with observed groundwater levels and publicly available streamflow and
climate data. The data set covers 23 groundwater sample sites and 39
surface water sites. Because the data was collected by four distinct
groups and over many years, we analyse to see if this has caused a bias
in the dataset. In addition, we show the major spatial and temporal
trends to provide an overview of the dataset. The dataset is made open
access to encourage further research and the current paper shows the
richness of the collected data and opportunities for further research.
\end{abstract}




\section{Introduction}

Dryland and irrigation salinity has long been a major natural resource
management concern in Australia
\citep{Jolly2001, White2009, Scanlon2007, Walker2002, Finlayson2010}.
Globally, the success of management of salinity, while extensively
documented, has remained patchy \citep{Leblanc2012}. As a result, in
Australia, the volume of research and number of publications in this
area has decreased significantly in recent years (Figure
\ref{fig:SalinityPapers}). This is partly due to the effect of the
millenium drought on groundwater levels and the consequent reduction in
the appearance of salinity effects in the landscape
\citep{Mcfarlane2016}. However, the reduction in research is also
because of the increased understanding that salinity processes are more
complex than previously recognised. For example, salinity processes can
vary substantially across the landscape \citep{Conyers2008} and the
processes of salt delivery to the stream also varies in the landscape
\citep{Summerell2006, Hughes2007} and is dependent on landscape
characteristics \citep{vanDijk2008, Dalhaus2010}. As a result, the
investment required to improve the understanding and increase the
effectiveness of management is considerable and this has resulted in a
reduction of the number of studies after the large investment in the
late 1990s and 2000s in Australia.

\begin{figure}
\includegraphics[width=0.9\linewidth]{Figures/Dryland Salinity Papers} \caption{Number of papers on the Web of Science related to the search terms (Dryland Salinity) AND Australia, 1980 - 2022}\label{fig:SalinityPapers}
\end{figure}

The poor spatial and timescale distribution of water quality datasets
has long been an obstacle to measuring trends in salinity. Studies such
as Jolly et. al. \citeyearpar{Jolly2001} and White et. al
\citeyearpar{White2009} used large historical datasets to detect
broadscale trends over large periods throughout the Murray Darling Basin
(MDB). This work identified that southern and eastern dryland regions in
the Murray Darling Basin have rising salinity trends that were worse in
areas of low rainfall \citep{White2009, Jolly2001}. However, the ion
composition varied greatly throughout the Murray Darling Basin (MDB)
\citep{White2009}. More specifically, Conyers et al.
\citeyearpar{Conyers2008} tried to isolate which areas in the middle
portion of the Murrumbidgee catchment acted as sources of salinity, as
well as whether this was predominantly marine cyclic salts (NaCl) as
previously assumed, or whether salts from mineral weathering were also
involved (e.g.~Ca, Mg, HCO\textsubscript{3}). While both can be measured
using conventional methods such as EC, a rise in marine cyclic salts can
be a major source of osmotic stress, whereas mineral weathering salts
are far less harmful and are more likely to precipitate at reasonably
low concentrations \citep{Conyers2008}. The ratio of
Cl:HCO\textsubscript{3} ions was identified as the best indicator of the
source of salinity, with Cl\textsuperscript{-} acting as a measure of
marine cyclic salts and HCO\textsubscript{3}\textsuperscript{-} acting
as a measure of mineral weathering salts. The Muttama catchment was
specifically identified as a candidate for future research as ion
concentrations appeared to result in a change from east to west,
correlating with the underlying geology and their mineral composition.

It is clear from these examples that detailed spatial and temporal
datasets are key to understanding different hydrogeochemical processes
in the landscape \citep[e.g.][]{Cartwright2010, Dalhaus2010}, but
overall publicly available datasets on dryland salinity in Australia
remain limited to detailed data from small experimental catchments
(\textless{} 100 ha) \citep{Summerell2006, Hughes2007} or sparse
government datasets from official monitoring
(i.e.~\href{https://waterinsights.waternsw.com.au/}{WaterNSW
WaterInsights platform}) which tend to be limited in hydrogeochemical
data. Part of this is related to the sensitivity of the data given the
relationship with possible land values. However, as the understanding of
salinity occurrence grows, this argument is less valid. Making data more
widely available would increase the opportunities for research and
increase our understanding of dryland salinity processes.

Without regular and expensive automated sampling, field campaigns to
collect water quality data tend to be ``snapshot'' activities
\citep{Grayson1997, Breuer2015, Lyon2008, Cartwright2010, Lintern2018}
which can be biased due to the over representation of low flow
conditions \citep{Lessels2020}. Even the analyses of substantial
government data bases \citep{Lintern2018} are likely to be biased in
this way. This means that overall there are limited streamflow and
groundwater salinity data sets that combine multiple locations across a
significant time period and that combine a range of flow
characteristics.

The aim of this paper is to present and describe the space time
dimensions and basic relationships of a complex groundwater and surface
water hydrogeochemistry dataset that was collected over a 14 year period
in a 1000 km\textsuperscript{2} agricultural catchment in New South
Wales, Australia. The Muttama catchment, which is the focus of this
paper, provides a microcosm of groundwater and surface water salinity
variability in Australia. Focusing on a medium size catchment in greater
detail creates opportunities to test whether sources of salinity can be
traced back to specific areas of land.

This paper gives a description of the dataset to facilitate open access
of the dataset, but does not analyse the physiochemical relationships in
the data in detail. This will be analysed in follow-up papers and was
partly analysed in an earlier thesis \citep{Akter2018}. The main aim of
this paper is to make the data set accessible to other researchers to
encourage further research in this catchment and in salinity in general.

\section{Methods}

\subsection{Muttama catchment}

The 1000 km\textsuperscript{2} Muttama creek catchment (Figure
\ref{fig:samplemap}) is located in the Mid-Murrumbidgee catchment area
of NSW in south eastern Australia. The landscape is undulating with
elevation variations ranging from 227 - 719 m. Muttama creek flows
north-south through the length of the catchment towards the Murrumbidgee
River near Gundagai. The main township, Cootamundra is located in the
upper half of the catchment. The dominant land use type of this
catchment is about 93\% agriculture, dominated by winter-spring cropping
and pasture. Mean annual rainfall (1891-2024) in the catchment is 654 mm
for the longest running Bureau of Meteorology Landgrove station (station
073022), while potential evapotranspiration far exceeds this total.

Streamflow is measured continuously by WaterNSW, the state agency
responsible for water data collection, at three locations in the
catchment: Coolac, station no. 410044, the main downstream point, and
Berthong, station no. 41000207 and Jindalee, station no. 410112 on two
branches above the Cootamundra township. This data is available as open
access via the \href{https://waterinsights.waternsw.com.au/}{WaterNSW
WaterInsights platform} or through the
\href{http://bom.gov.au/waterdata/}{Bureau of Meteorology}. Here, we
only use the data from the Coolac station as a comparison.

\begin{figure}
\includegraphics[width=0.8\linewidth]{Figures/gw_or_sw_map} \caption{Muttama Catchment Sampling Locations with Elevation. Symbol colour indicates whether location was a groundwater or surface water source, with blue being surface water and brown/orange being groundwater. The numbers on the map represent the sample location number.}\label{fig:samplemap}
\end{figure}

The depth to the nearest groundwater table varies across the Muttama
catchment and it ranges from \textless{} 2 m to 20 m below ground level
\citep{DECC2009}. Deep groundwater in the catchment occurs mostly in
fractured rock aquifers common on the eastern, and western fringes of
the catchment. In contrast, shallow groundwater is associated with
unconfined alluvial, colluvial, and eluvial aquifers. Some aquifers in
the northern part of this catchment show artesian behavior
\citep{Webb1999, Akter2018}.

Saline areas of the catchment tend to be associated with geological
heterogeneity, primarily the sedimentary materials in the west and
rhyolite on the northwest side \citep{Conyers2008}. Overall, Muttama
creek is a significant salt contributor to the downstream Murrumbidgee
river with suggested contributions of around 58\% from cyclic sources
and 42\% salts originating from mineral weathering \citep{Conyers2008}.

\subsection{Data Collection}

\subsubsection{Data Sources}

The water quality dataset contains data from 4 main sampling sources
related to four distinct groups of ``people'' doing the sample
collection. The term ``people'' is used loosely, as it mainly related to
four different types of sampling campaigns, which potentially had
differences in the rigour of the sampling campaign (quality control,
types of samples taken, training of the people taking the samples).
These groups are designated as:

\begin{itemize}
\item
  Source 1: Data from the PhD study by Akter \citeyearpar{Akter2018}.
\item
  Source 2: Data from the sampling campaign of two former students, the
  PhD from Lessels \citeyearpar{Lessels2014} and unpublished data from
  another student, E. Milne.
\item
  Source 3: A dataset collected by undergraduate and postgraduate
  students as part of field trips in different units of study at the
  University of Sydney is identified as ``Student data''. This data was
  sampled ``ad-hoc'' during the field trip period using standard
  sampling protocols as described for the data from Akter
  \citeyearpar{Akter2018}.
\item
  Source 4: Data from several autosamplers installed in the catchment
  during the PhD from Lessels \citeyearpar{Lessels2014}. Because these
  samples were not taken by a ``person'' and were taken on a flow
  weighted basis, we separated the data from the ``grab'' samples in the
  other methods. These samples are also missing field measurements, as
  these were only analysed in the laboratory.
\end{itemize}

Overall, 1160 water samples were collected from 62 sample locations over
the 2010 - 2024 period. However, not all sites were sampled at all times
and not all samples were fully analysed for all hydrogeochemical
variables. Both surface water and groundwater samples were collected at
23 groundwater sample sites and 39 surface water sites. These are
distributed across the catchment, depending on standing water
availability and access.

In addition to the water quality dataset, data from 23 groundwater data
loggers is provided from the same groundwater sample sites as in the
hydrogeochemical dataset.

\subsubsection{Hydrogeochemical Variables}

The overall structure of the hydrogeochemical dataset consists of
repeated measurements over time at multiple locations in Muttama
catchment. For each location the name of the location and the spatial
coordinates were recorded in decimal degrees (Longitude = x and Latitude
= y) as well as whether the location was a groundwater or a surface
water location. The names of the locations are fairly random and basic
locality indicators, which cannot be interpreted exactly.

The data for each location consist of up to six variables which were
measured in the field (Table \ref{tab:TableMeasurements}). These were
complemented by laboratory analysis, which repeated some of the field
measurements, and for additional major anion and cation variables (Ca,
Mg, Na, K, Cl, SO\textsubscript{4}, HCO\textsubscript{3}), and total
Nitrogen (N) and total Phosphorus (P) for part of the sample set. Some
other variables were infrequently measured and are not included in the
data set.

\begin{table}
\centering
\caption{\label{tab:TableMeasurements}Variables measured in the field and laboratory.}
\centering
\begin{tabular}[t]{l|l|l|l|l}
\hline
Field measurements & Lab repeat & Anions & Cations & Others\\
\hline
pH & pH & Cl$^-$ & Na$^+$ & Total Nitrogen\\
\hline
EC (Electrical conductivity) & EC & HCO$_3^-$ & Mg$^{2+}$ & Total Phosphorus\\
\hline
SPC (temperature corrected EC) & SPC & SO$_4^{2-}$ & K$^+$ & \\
\hline
Temperature &  &  & Ca$^{2+}$ & \\
\hline
Alkalinity (HCO$_3^-$) &  &  &  & \\
\hline
Dissolved Oxygen (DO) &  &  &  & \\
\hline
Turbidity &  &  &  & \\
\hline
\end{tabular}
\end{table}

The variables pH, EC, SPC (specific conductance: field temperature
corrected EC), Temperature, and in some cases DO and Turbidity were
measured using a range of field probes. All field probes measured pH,
EC, Temperature and calculated SPC. Early measurements (samples up to
November 2014) used a YSI probe that included a turbidity and DO probe
(YSI 6600 and YSI 600 for surface and groundwater, respectively). Later
groundwater samples (After November 2014) used a different YSI probe
(YSI ProPlus multi-parameter) that only included a DO probe. Finally,
(after mid 2019) surface water and groundwater sampling used a Xylem Exo
probe with DO, pH, EC, Temperature and SPC. Some of the variability in
the field measurements might be due to this variation in the field
instrumentation, because the exact instrument information was not
recorded with the data.

Anions in most of the samples were analysed using high-performance
liquid chromatography method (Dionex P680 HPLC) and cations were
measured on acidified samples using an Inductively Coupled Plasma
Optical Emission Spectrometer (ICP-OES, Varian 720-ES) at the University
of Sydney \citep{Akter2018}. Duplicate samples in the analysis had a
reported relative percentage difference (RPD) lower than 5\% in part of
the sample dataset \citep{Akter2018}. Some of the later samples and the
student sample set (Source 3) were analysed by a commercial laboratory
(\href{https://www.alsglobal.com/en/locations/asia-pacific/pacific/australia/nsw/sydney-woodpark-environmental}{ALS
Environmental, Smithfield, NSW}). Alkalinity concentrations were
generally measured in the field within 24h of collection using a HACH
digital titrator (model 16900) \citep{Akter2018} up to 2017 and by the
commercial laboratory after this.

The hydrogeochemical data is stored on the University of Sydney
escholarship repository: \url{doi.org/10.25910/m0wp-8890}

\subsection{Continuous variables}

The logger data, which collected groundwater pressure levels at 15 min
and later at 2 hour intervals, were adjusted for the length of the cable
and the height of the standpipe above the ground level. They were
subsequently summarised to raw daily values using an R script
(\texttt{SummariseDailyData.R}), which is stored with the raw data in
the Open Science Foundation (OSF) repository associated with this paper
\url{https://doi.org/10.17605/OSF.IO/BEUWK}.

The loggers in the field were uncalibrated. Due to logger failures, gaps
occur in the daily data, followed by replacement of the faulty loggers.
In some cases the cable length was adjusted and this was recorded in the
field notes. Overall, this resulted in data with gaps and sometimes
shifts in the recorded logger data.

Manual water level measurements were taken at each manual sampling date
to allow calibration of the logger data. To correct the groundwater
logger data, the daily data was matched to the observed data using
linear regression, if more than 3 manual observed data were available
and slope and intercept of the regression had a p-value \textless{} 0.1.
If there were less than 3 manual observed data points for the specific
logger an adjustment to the data was based on the difference between the
average observed data and the average recorded water levels. Otherwise
no adjustment was made. This is slightly tricky: After the manual
observations are made, the well is purged and the logger is temporarily
removed from the well. This data (during the temporarily removal of the
logger during the purging and subsequent recovery of the groundwater
level) is removed from the logger data series. As a result, there is no
direct time match between the logger data series and the manual
observations. However, the data showed that, in most cases, the
groundwater level recovered within 24 hrs.

To explain this more clearly the following pseudo code describes the
process:

\begin{algorithm}
\caption{Pseudo code cleaning groundwater level data}
\label{a1}
\begin{algorithmic}
\IF {sufficient points \> 3} 
        \STATE run a regression between interpolated depth and observed depth  
        \STATE check the p-values of the slope and intercept
        \IF {the slope is not significant (using p \> 0.10)}
                \STATE  use only the intercept to correct the logger data
        \ELSE
                \IF {the intercept is not significant, but the slope is}
                        \STATE use only the slope to correct the logger data
                \ELSE
                       \STATE use both slope and intercept of the regression to correct the logger data
                \ENDIF
        \ENDIF
\ELSE
        \STATE There are not enough values for a regression, use mean difference between logged and observed value to correct the logger data
\ENDIF
\end{algorithmic}
\end{algorithm}

The code used to match the manually observed data with the logger data
is in the script \texttt{Match\_obs\_logger\_data.R}, which is stored
with the raw data in the Open Science Foundation (OSF) repository
associated with this paper \url{https://doi.org/10.17605/OSF.IO/BEUWK}.

After the automated process, two of the groundwater level data series
still had substantial discrepancies in some sections of the data. This
was most likely due to the a lack of observed data for the specific
logger. A final manual correction was applied. As this process is based
on judgement of the data by the authors of this paper, we documented
this in detail in the supplementary material.

The final corrected data that is published with this paper on
\url{https://doi.org/10.17605/OSF.IO/BEUWK} includes a column which
describes whether the data is based on the automatic correction or a
further manual correction.

\subsection{Boxplots and maps}

Using the most complete data, boxplots and spatial maps were generated
to highlight the spatial and temporal variation in the data set. The
boxplots visually highlight empirical differences between data
variables.

The mean concentration and interquartile range (25\textsuperscript{th} -
75\textsuperscript{th} percentile) of the concentration data
distributions were calculated to give an indication of variation of the
data in the spatial maps.

All graphs and maps were produced using R version 4.3.1 \citep{R2023}.
All code can be found in the associated
\href{https://github.com/WillemVervoort/MuttamaDataPaper}{github
repository}, as part of the markdown document for this paper.

\section{Results}

\subsection{Distribution of missing Values}

\begin{figure}
\includegraphics[width=0.9\linewidth]{Figures/na_count} \caption{Distribution of missing values for the different data sources and measurement types. Most of these missing values were because not all the variables were analysed for all the samples, see the explanation in the article text}\label{fig:na-plot}
\end{figure}

\clearpage

\begin{figure}
\includegraphics[width=0.9\linewidth]{Figures/na_GW} \caption{Percent missing values for the groundwater data across all piezometers. Thicker blue lines mean a group of missing values on closely related dates}\label{fig:gw-na-plot}
\end{figure}

In the hydrogeochemistry data, data source 3 was the most complete in
terms of variables analysed, because in this set more variables were
analysed in the commercial lab (Fig \ref{fig:na-plot}). Some of the
variables analysed in the commercial lab were not analysed with the
equipment at the University of Sydney. However, source 3 had the
smallest number of overall samples. Source 2 has the most incomplete
data points. Source 4 has a very consistent number of missing values,
possibly because not all samples were analysed in the set. For source 2,
the missing data suggests that for many of the samples only a few of the
variables were measured and analysed as highlighted above. In the data
from source 2, almost 50\% of samples are incomplete in terms of the
measurement of all the variables in Table 2. Similarly, source 1 had
more incomplete data, because some of the minor elements were not
analysed. The field recorded variables pH, EC, SPC and Temperature were
the most complete as they were generally measured directly in the field.
Thus the distribution of the NA values in the overall data set is mostly
a reflection of the time period of sampling and the change in
methodology over the 14 years of sampling.

In the groundwater level time series, the missing data relate mostly to
logger failures and the different times that wells were instrumented.
Rather than giving a full breakdown by well location, the overall level
of completeness of the series is displayed (Fig \ref{fig:gw-na-plot}).

\subsection{Temporal Distribution of Data}

\clearpage

\begin{figure}
\includegraphics[width=0.5\linewidth]{Figures/monthly} \caption{Distribution of samples throughout the year (total number of samples collected during each month) against average monthly rainfall during the study period. }\label{fig:month-plot}
\end{figure}

Overall the water quality sampling appears to have a reasonable
distribution across all months (Fig \ref{fig:month-plot}), therefore
seasonal trends should be identifiable in the data. Average rainfall
data (1995 - 2022) does not indicate any major seasonal trends, although
there is a slight dominance of rainfall in the early Austral Spring
(months 9 and 10, September and October). This also explains the higher
number of samples because, more sampling trips were organised in this
period, and spatially more channels could be sampled for surface water.
The timing of source 3 (the student data) is the result of the yearly
field trips, which tended to occur at approximately the same time of the
year in early March and late September or early October, coinciding with
the University semesters.

\begin{figure}
\includegraphics[width=0.5\linewidth]{Figures/annual} \caption{Distribution of samples over the study period (total number of samples collected during each year) against average monthly rainfall during the study period. }\label{fig:annual-plot}
\end{figure}

The number of samples collected in each year relative to the different
sources changes throughout the years reflecting the duration of the
different studies and funding cycles (Fig \ref{fig:annual-plot});
however the data is still well-distributed enough that overall trends
should be clear. In addition, some of the sample volumes can be related
to the occurrence of rainfall, as in drier years several of the channels
would be dry and no sampling of surface water could occur.

Consistent groundwater sampling commenced later in the project, which
means that there are very few groundwater samples before 2013. In
contrast, the autosamplers were installed early in the project and there
are no samples from this source after 2013.

\begin{figure}
\includegraphics[width=0.8\linewidth]{Figures/FDC} \caption{Sample distribution on flow duration curve derived from flow data at the Coolac NSW government station.}\label{fig:FDC}
\end{figure}

Surface water samples were reasonably well distributed across the flow
distribution, measured at the Coolac station (410044), with only a
possible bias towards periods of medium flow. This is most likely since
many of the upstream surface water sampling points are often completely
dry during periods of low flow at Coolac, and therefore cannot be
sampled. Conversely, there are no manual samples during high or very
high flow as during flood situations sampling was dangerous and
restricted by work health and safety considerations. The samples at high
flow are all from our automated sampling.

\subsubsection{Comparisons of Groundwater samples with Surface water
samples}

\begin{table}
\centering
\caption{\label{tab:TableElementstats}Summary statistics for elements measured in the field}
\centering
\begin{tabular}[t]{l|l|l|l|l|l|l}
\hline
\multicolumn{1}{c|}{} & \multicolumn{3}{c|}{GW} & \multicolumn{3}{c}{SW} \\
\cline{2-4} \cline{5-7}
Element & Mean & Min & Max & Mean & Min & Max\\
\hline
Temperature field & 17.4 & 11.4 & 30.8 & 15.9 & 4.1 & 32.9\\
\hline
Turbidity field & 30.2 & 0.1 & 90.0 & 13.1 & 0.5 & 51.6\\
\hline
DO field [\%] & 20.1 & -3.2 & 70.7 & 86.8 & 10.4 & 220.2\\
\hline
EC field [µS cm$^{-1}$] & 4006 & 457 & 14385 & 1411 & 117 & 4163\\
\hline
EC lab [µS cm$^{-1}$] & NA & NA & NA & 1292 & 1 & 5230\\
\hline
SPC field & 4667 & 347 & 17799 & 1246 & 61 & 6046\\
\hline
pH field & 7.3 & 6.0 & 8.8 & 8.0 & 6.7 & 10.0\\
\hline
pH lab & NA & NA & NA & 7.6 & 6.8 & 8.5\\
\hline
Cl$^-$ [mg L$^{-1}$] & 1218.3 & 73.0 & 4960.7 & 294.2 & 8.9 & 4577.1\\
\hline
HCO$_3^-$ [mg L$^{-1}$] & 599.5 & 68.0 & 1115.0 & 333.9 & 25.6 & 782.0\\
\hline
SO$_4^{2-}$~[mg L$^{-1}$] & 261.1 & 0.0 & 1319.2 & 48.4 & 0.0 & 987.6\\
\hline
Na$^{+}$ [mg L$^{-1}$] & 492.1 & 17.8 & 2283.7 & 122.1 & 7.1 & 866.2\\
\hline
Mg$^{2+}$ [mg L$^{-1}$] & 167.6 & 6.4 & 654.6 & 66.7 & 1.4 & 276.9\\
\hline
K$^{+}$ [mg L$^{-1}$] & 4.4 & 0.0 & 30.8 & 7.6 & 1.0 & 43.8\\
\hline
Ca$^{2+}$ [mg L$^{-1}$] & 256.4 & 16.3 & 1354.2 & 61.6 & 2.3 & 682.1\\
\hline
TN [mg L$^{-1}$] & 3.7 & 0.6 & 10.3 & 1.2 & 0.0 & 4.9\\
\hline
TP [mg L$^{-1}$] & 0.3 & 0.0 & 1.0 & 0.5 & 0.0 & 5.5\\
\hline
\end{tabular}
\end{table}

The summary of the samples (Table \ref{tab:TableElementstats})
highlights the range of the data for the different variables. Obviously,
surface water will record higher DO values, while groundwater recorded
higher EC and SPC values. The rest of the variables have fairly similar
ranges for both groundwater and surface water. Both total P and total N
are low across the catchment samples, with only a few outliers related
to specific locations and dates.

Groundwater samples have quite a distinctive hydrogeochemical signature
compared to the surface water samples (Fig \ref{fig:gw_sw-plot}). Since
`Source 1' collected most of the groundwater samples, this results in
differences between data collection sources. Field SPC measurements were
used to represent EC since these samples had the fewest missing data.
There also appears to be a slight bias towards lower EC values for
sampling group 2, but this is likely because these samples were
collected during two very wet years in 2010 and 2011 (Fig
\ref{fig:annual-plot}) and are mostly associated with high flow values
(Fig \ref{fig:FDC}).

\begin{figure}
\includegraphics[width=0.8\linewidth]{Figures/gwsw} \caption{Difference in pH and EC for groundwater and surface water samples. The source of the data (the sampling group) is indicated with colour.}\label{fig:gw_sw-plot}
\end{figure}

\subsubsection{Groundwater level data}

\begin{figure}
\includegraphics[width=1\linewidth]{Figures/Final_Corrected_piezodepths} \caption{Overview of the corrected groundwater time series for all the wells. Different panels relate to the different locations highlighted in Figure 2.}\label{fig:gw-series}
\end{figure}

The overall corrected groundwater timeseries shows the shorter time that
loggers were installed in the wells (Fig \ref{fig:gw-series}),
associated with the PhD thesis from Akter \citeyearpar{Akter2018}. It
also indicates that the manual data can not always be fully matched with
the logger data, but further corrections are likely to be speculation.

In general, shallow groundwater occurs between 1 and 5 meters below the
surface and is responsive to dry and wet periods. Some of the wells have
positive pressures, resulting in occasional groundwater levels above the
ground surface, such as at GW10, GW12, GW13 and GW21 (Fig
\ref{fig:gw-series}).

\subsection{Spatial variation}

\begin{figure}
\includegraphics[width=1\linewidth]{Figures/ec_map} \caption{Spatial Variation of EC throughout the catchment, using Mean EC in $\mathrm{\mu S~cm^{-1}}$ and interquartile range (IQR) for each sampling location. Only locations with more than 10 observations over the 14 years are included.Surface water sample sites are on the left panel, while groundwater sample sites are on the right panel. }\label{fig:ECmap}
\end{figure}

\clearpage

\begin{figure}
\includegraphics[width=0.8\linewidth]{Figures/ec_plot} \caption{Variation in EC throughout the catchment across time by sample location presented as boxplots. Only locations with more than 10 observations over the 14 years are included. The blue dashed line is the drinking water limit of 800 $\mathrm{\mu S~cm^{-1}}$.}\label{fig:ECboxplot}
\end{figure}

\clearpage

\begin{figure}
\includegraphics[width=0.8\linewidth]{Figures/clhco3_map} \caption{Spatial Variation of $\mathrm{Cl^-:HCO_3^-}$ ratio throughout the catchment, highlighting mean and interquartile range (IQR) for each sampling location. Only locations with more than 10 observations over the 14 years are included. Surface water sample sites are on the left panel, while groundwater sample sites are on the right panel.}\label{fig:Carbonate-map}
\end{figure}

\clearpage

\begin{figure}
\includegraphics[width=0.8\linewidth]{Figures/clhco3_plot} \caption{Variation across time of the $\mathrm{Cl:HCO_3}$ ratio by sampling location represented as a boxplot. Only locations with more than 10 observations over the 14 years are included. The blue dashed line indicate a $Cl:HCO_3$ ratio of 1}\label{fig:Carbonate-boxplot}
\end{figure}

There is clear spatial variation in water parameters throughout the
catchment, including between groundwater and surface water sampling
sites (Figures \ref{fig:ECmap} and \ref{fig:Carbonate-map}). As
examples, the spatial distributions for EC and Cl:HCO\textsubscript{3}
are shown for sample sites with more than 10 observations over the
sampling period. Similar maps can be easily generated for other
parameters using the code in the markdown document. In the map, the
concentration is indicated by the size of the symbol, while the colour
shading indicates the variability. This suggests surface water samples
had lower variability and lower salt concentrations. In addition,
samples on the North western side of the catchment had higher salt
concentrations and higher Cl:HCO\textsubscript{3} ratios, which is also
associated with higher variance in the samples.

Below the maps, boxplots (Figures \ref{fig:ECboxplot} and
\ref{fig:Carbonate-boxplot}) highlight the difference in the
distributions between the surface water sample sites and the groundwater
sample sites. For sites that have more than 10 observations, this
highlights the difference between sample sites, reflected spatially on
the maps. For example, it highlights that high EC sites also tended to
have high Cl:HCO\textsubscript{3} values, and conversely that low
salinity groundwater tended to be associated with low
Cl:HCO\textsubscript{3} values. It also points out the single well
(GW23) that has a very low EC. Previous studies have suggested there may
be a difference in Cl:HCO\textsubscript{3} ratio in surface water
between the eastern and western parts of the Muttama catchment
\citep{Conyers2008}, and Figure \ref{fig:Carbonate-map} suggest a
similar pattern, with samples in the North and West being higher in EC
and high in Cl:HCO\textsubscript{3} values, while the Eastern and
Southern areas have lower variability in the surface water samples and
lower Cl:HCO\textsubscript{3} values.

\begin{figure}
\includegraphics[width=0.8\linewidth]{Figures/piper_plot} \caption{Piper-plot of all groundwater and surface water data collected over the period wiht complete major anion data}\label{fig:piperplot}
\end{figure}

\subsection{Piper plot}

The piper plot contains the data for all the samples with complete major
anion and cation data, which are 500 samples. Of these, 229 are surface
water samples and the rest groundwater samples. The piper plot of the
data (Figure \ref{fig:piperplot}) does not provide much clarity as the
samples cover a large area across the ternary space. There is a slight
shift towards the HCO\textsubscript{3} and Ca/Mg type waters for the
groundwater samples compared to the surface water samples, which are
more Na dominated. There is also a small cluster of surface water
samples that are more SO\textsubscript{4} dominated, potentially
indicating different geological origins as mentioned earlier in the
paper. However, the some of the groundwater samples are very high in Cl,
explaining the high EC values observed for these samples.

\section{Discussion}

The dataset in this paper is unique in Australia. There are substantial
datasets from experimental small catchments
\citep[i.e.][]{Hughes2007, Summerell2006}, but not many of these are
publicly or easily accessible. In contrast, there are data from very
large state and national datasets \citep[i.e.][]{Jolly2001}, but there
are limited publicly available data sets that cover substantial space
and time scales.

The dataset presented here covers multiple sites, multiple time periods
and multiple sampling campaigns. This is a strength, but also has
limitations. As the groundwater data clearly shows, the actual number of
possible sample sites is limited by the existing and accessible
groundwater wells. Over the 14 years of research, several groundwater
wells were accidentally destroyed during farm operations, further
reducing the sample opportunities. The number of surface water sampling
sites in the catchment is limited by the ephemeral nature of the stream
network, with some creeks not flowing for long periods. Given the travel
distance from the University of Sydney, this also meant that overall
sampling is limited in scope, as is clearly shown in the spatial maps
(Figures \ref{fig:ECmap} and \ref{fig:Carbonate-map}).

The development of the dataset over many years regrettably does not
allow a full uncertainty analysis. There is reasonable quality
assessment of the laboratory analysis of the samples taken by
\citet{Akter2018} (see the appendices in \citet{Akter2018}), there is
less reporting of this for the other samples. An exception to this is
the samples analysed by the commercial laboratory, where a strict
quality protocol was followed. Some uncertainty can be gleaned from the
mass balance closure of the major cations for the laboratory analysis.
None of this provides an understanding of the uncertainty associated
with the field measurements and the potential manual handling errors.
Despite these potential sources of uncertainty, the analysis in this
paper highlights the spatial and temporal consistency of the data, thus
providing evidence of manageable uncertainty in overall dataset.
Therefore, the are valuable in the general space time information that
it provides.

The dataset suggests implications for management of salinity in the
Muttama catchment. The samples clearly indicate that the main source of
high Cl salinity originates from the sediments and rocks on the
northwestern side of the catchment. This is further complicated by the
artesian nature of some of the wells in this area. However, as the water
level data and \citet{Akter2018} has highlighted, the shallow
groundwater levels only partly respond to rainfall recharge. Investment
and incentives to reduce recharge should focus on these areas,
particularly to limit the movement of saline discharge into the creek
\citep{Akter2018}.

\section{Conclusions}

Detailed datasets for medium to large scale catchments are
underrepresented in salinity research in Australia and worldwide. This
paper reports on a long term (14 year) hydrogeochemistry dataset from a
single medium scale catchment (\textgreater{} 1000
km\textsuperscript{2}) in NSW, Australia. This dataset includes a total
of 1160 water samples from 62 locations within the catchment and
includes groundwater and surface water sites. While the dataset was
collected by different groups of people at different times and
locations, it still provides a valuable long term and spatially diverse
data set. The complete sample set covers a wide range of flow and
wetness conditions. Clear differences were observed in pH,
electroconductivity, and in ion ratios between groundwater and surface
water samples. Spatial differences in water chemistry are also apparent,
with our data reinforcing prior chemical gradients observed in the
catchment. Though in this paper we only provide a limited analysis of
the data, we anticipate the dataset can be used for research to gain
better insight into the spatio-temporal evolution of hydrogeochemical
processes from larger catchments and improve the understanding of
catchment salinity.



\codedataavailability{The majority of the code and associated data,
including the Rmarkdown for this paper, is stored on Github
\url{https://github.com/WillemVervoort/MuttamaDataPaper}.\textbackslash{}
The hydrogeochemistry data is located on
\href{doi.org/10.25910/m0wp-8890}{the University of Sydney escholarship
repository}.\textbackslash{}\\
However, due to the volume of raw data, the groundwater logger data is
stored in a separate Open Science Foundation project:
\url{https://doi.org/10.17605/OSF.IO/BEUWK}} %% use this section when having data sets and software code available



%%%%%%%%%%%%%%%%%%%%%%%%%%%%%%%%%%%%%%%%%%
%% optional

%%%%%%%%%%%%%%%%%%%%%%%%%%%%%%%%%%%%%%%%%%

%%%%%%%%%%%%%%%%%%%%%%%%%%%%%%%%%%%%%%%%%%
\authorcontribution{T. Bishop initiated the sampling campaign. T.
Bishop, F. van Ogtrop, F. Dijkstra and R.W. Vervoort conceptualised the
overall study, and managed the project. R.W. Vervoort and M. Tambrchi
wrote the draft paper and analysed results. F.Akter and J. Lessels
collected and analysed the majority of the samples. M. Tambrchi, A.
Buzacott, J. Moloney, F. Akter analysed and managed the data. All
authors participated in sampling, laboratory analysis and review of the
paper.} %% optional section

%%%%%%%%%%%%%%%%%%%%%%%%%%%%%%%%%%%%%%%%%%
\competinginterests{The authors declare no competing
interests.} %% this section is mandatory even if you declare that no competing interests are present

%%%%%%%%%%%%%%%%%%%%%%%%%%%%%%%%%%%%%%%%%%

%%%%%%%%%%%%%%%%%%%%%%%%%%%%%%%%%%%%%%%%%%
\begin{acknowledgements}
This dataset would not have been possible without the generous
assistance and access to properties from the following land owners in
the Muttama Creek area: R. Last, the Tozer family, M. Sullivan, P.
McClintock, P. McGuire, S. Sharman, A. Hollihan, the managers at Brawlin
Springs as part of the Romani Pastoral Company, and the managers and
owners of Wavehill, in particular J. Litchfield. We would like to thank
several generations of students in the units LWSC2002, ENVX3003 and
ENSY5708, as well a multiple interns from French Institutions and
University and Wageningen University for assisting with the sample
collection.
\end{acknowledgements}

%% REFERENCES
%% DN: pre-configured to BibTeX for rticles

%% The reference list is compiled as follows:
%%
%% \begin{thebibliography}{}
%%
%% \bibitem[AUTHOR(YEAR)]{LABEL1}
%% REFERENCE 1
%%
%% \bibitem[AUTHOR(YEAR)]{LABEL2}
%% REFERENCE 2
%%
%% \end{thebibliography}

%% Since the Copernicus LaTeX package includes the BibTeX style file copernicus.bst,
%% authors experienced with BibTeX only have to include the following two lines:
%%
\bibliographystyle{copernicus}
\bibliography{datapaper.bib}
%%
%% URLs and DOIs can be entered in your BibTeX file as:
%%
%% URL = {http://www.xyz.org/~jones/idx_g.htm}
%% DOI = {10.5194/xyz}


%% LITERATURE CITATIONS
%%
%% command                        & example result
%% \citet{jones90}|               & Jones et al. (1990)
%% \citep{jones90}|               & (Jones et al., 1990)
%% \citep{jones90,jones93}|       & (Jones et al., 1990, 1993)
%% \citep[p.~32]{jones90}|        & (Jones et al., 1990, p.~32)
%% \citep[e.g.,][]{jones90}|      & (e.g., Jones et al., 1990)
%% \citep[e.g.,][p.~32]{jones90}| & (e.g., Jones et al., 1990, p.~32)
%% \citeauthor{jones90}|          & Jones et al.
%% \citeyear{jones90}|            & 1990


\end{document}
